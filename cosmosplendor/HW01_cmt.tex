\documentclass[10pt]{article}
\usepackage{kotex}
\usepackage{mathpazo}
\usepackage{graphicx} % Required for inserting images
\usepackage[linguistics]{forest}
\usepackage[a4paper]{geometry}
\usepackage[T1]{fontenc}
\usepackage{titlesec}
\usepackage{titling}
\usepackage{setspace}
\usepackage{titlesec}
\usepackage{microtype}

\setlength{\droptitle}{-4em}

\makeatletter
\titleformat{\subsubsection}{\scshape\bfseries}{\thesection}{1em}{}
\makeatother

\geometry{
 a4paper,
 left=30mm,
 right=30mm,
 top=30mm,
 bottom=30mm
 }

\forestset{default preamble={for tree={s sep=5mm, inner sep=2, l=0}}}

\title{2024-2_Syntax_HW01}
\author{이우성}
\date{2024.10.10}

\begin{document}
% --------------------------------------------------------------
%                         Start here
% --------------------------------------------------------------
\title{2024-2 통사론 HW\#01 - commented}
\author{언어학과 22 이우성}
\maketitle
\text{}
\newline
\subsubsection*{\textls[-50]{Q1  GPS8. Ambiguiuty II}}
a) Alvina said Moises went to the store qiuickly.\\
\newline
1) \textit{"Moises went to the store qiuickly," Alvina said.}\\
\begin{forest}
[TP[NP[N\\Alvina]][VP[V\\said][CP[TP[NP[N\\Moises]][VP[V\\went][PP [P\\to][NP[D\\the][N\\store]]][AdvP[Adv\\quickly]]]]]]]
\end{forest}
\\
\newline
2) \textit{"Moises went to the store," Alvina said quickly.}\\
\begin{forest}
    [TP[NP[N\\Alvina]][VP[V\\said][CP[TP[NP[N\\Moises]][VP[V\\went][PP [P\\to][NP[D\\the][N\\store]]]]]][AdvP[Adv\\quickly]]]]
\end{forest}
\newpage
\noindent
d) Enraged cow injures farmer with ax\\
\\
3) \textit{Using an ax, an enraged cow injures a farmer.}\\
\begin{forest}
    [TP[NP[AdjP[Adj\\Enraged]][N\\cow]][VP[V\\injures][NP[N\\farmer]][PP[P\\with][NP[N\\ax]]]]]
\end{forest}
\\
4) \textit{A farmer with an ax is injured by an enraged cow.}\\
\begin{forest}
    [TP[NP[AdjP[Adj\\Enraged]][N\\cow]][VP[V\\injures][NP[N\\farmer][PP[P\\with][NP[N\\ax]]]]]]
\end{forest}
\subsubsection*{\textls[-50]{Q2  CPS7. Using Constituency Tests}}
\begin{spacing}{1.5}
    \noindent
    a) Juliet says that Romeo lies to his parents a lot.\\
    (a)는 [$_{\text{AdvP}}$ [$_{\text{Adv}}$a lot]] 이 수식하는 VP의 핵이 무엇인지에 따라 다른 의미로 해석될 수 있다.\\
    5) \textit{Juliet says a lot that Romeo lies to his parents.}
\end{spacing}
\noindent
\begin{forest}
    [TP[NP[N\\Juliet]][VP[V\\says][CP[C\\that][TP[NP[N\\Romeo]][VP[V\\lies][PP[P\\to][NP[D\\his][N\\parents]]]]]][AdvP[Adv\\a lot]]]]
\end{forest}
\newpage
\begin{spacing}{1.3}
    \noindent
    6) \textit{"Romeo lies to his parents a lot," Juliet says.}
\end{spacing}
\noindent
\begin{forest}
    [TP[NP[N\\Juliet]][VP[V\\says][CP[C\\that][TP[NP[N\\Romeo]][VP[V\\lies][PP[P\\to][NP[D\\his][N\\parents]]][AdvP[Adv\\a lot]]]]]]]
\end{forest}\\
\noindent
\newline
\begin{spacing}{1.3}
    \noindent
    [$_{\text{AdvP}}$ [$_{\text{Adv}}$a lot]] 이 says 를 수식하는 경우, says를 핵으로 갖는 VP의 주체인 Juliet이 어떠한 발언을 많이 한다는 의미로 해석된다. 한편 [$_{\text{AdvP}}$ [$_{\text{Adv}}$a lot]] 이 lies 를 수식하면 주체는 Romeo가 되어 그가 거짓말을 많이 했다고 Juliet이 말한 것으로도 해석할 수 있다.\\
    만일 아래와 같이 VP-preposing을 통해 [$_{\text{AdvP}}$ [$_{\text{Adv}}$a lot]]이 수식하는 동사를 단 하나로 제한해 보일 수 있다면 이러한 중의성은 해소된다.
\end{spacing}
\noindent
\begin{spacing}{1.3}
    \noindent
    c) Lies to his parents a lot, Juliet says that Romeo does.\\
\end{spacing}
\noindent
\begin{forest}
    [VP$_{\text{i}}$[V\\lies][PP[P\\to][NP[D\\his][N\\parents]]][AdvP[Adv\\a lot]]]
\end{forest} \qquad \begin{forest}
    [TP[NP[N\\Juliet]][VP[V\\says][CP[C\\that][TP[NP[N\\Romeo]][VP[V$_{\text{i}}$\\does]]]]]]
\end{forest}\\
\newline
\begin{spacing}{1.3}
    \noindent
    그러나 다음과 같이 VP-preposing을 하더라도 [$_{\text{AdvP}}$ [$_{\text{Adv}}$a lot]]이 prepose하려는 VP 안에 modifier로서 포함되지 못하는 경우에는 (a)와 같은 이유로 문장이 여전히 중의적인 의미를 갖게 된다.\\
    \textcolor{red}{
        b) Eat apples, Julian does everyday.\\
        1. (b) 예시를 통해 VP-constituent에서 AdvP를 떼어내고 preposing 할 수 있음을 파악해야 함.\\
        2. a lot이 says를 수식하는 경우에 이동이 불가능한 이유도 구체적으로 서술해야 함. (-1)\\}
\end{spacing}
\newpage
\begin{spacing}{1.3}
    \noindent
    d) Lies to his parents, Juliet says that Romeo does a lot.
\end{spacing}
\begin{forest}
    [VP$_{\text{j}}$[V\\lies][PP[P\\to][NP[D\\his][N\\parents]]]]
\end{forest}\\
    \begin{forest}
        [TP, s sep=3mm [NP[N\\Juliet]][VP[V\\says][CP[C\\that][TP[NP[N\\Romeo]][VP[V$_{\text{j}}$\\does,name=tgt,tikz={\node [draw,ellipse,inner sep=-1pt, fit to=tree]{};}][,phantom][AdvP,name=src,tikz={\node [draw,ellipse,inner sep=-1pt, fit to=tree]{};}[Adv\\a lot]]]]]]]
        \draw[->,>=stealth] (src) to[out=north east,in=north west] node[pos=0.5,yshift=3mm,xshift=15mm]{\textit{modifies}} (tgt);
    \end{forest}
    \begin{forest}
        [TP, s sep=3mm [NP[N\\Juliet]][VP[V\\says,name=tgt,tikz={\node [draw,ellipse,inner sep=-1pt, fit to=tree]{};}][CP, l=10mm [C\\that][TP[NP[N\\Romeo]][VP[V$_{\text{j}}$\\does]]]][AdvP,l=10mm,name=src, tikz={\node [draw,ellipse,inner sep=-1pt, fit to=tree]{};}[Adv\\a lot]]]]
        \draw[->,>=stealth] (src) to[out=north,in=north east] node[pos=0.5,yshift=3mm,xshift=15mm]{\textit{modifies}} (tgt);
    \end{forest}
\subsubsection*{\textls[-50]{Q3  CPS8. Leash and Clean up after your pet.}}
\begin{spacing}{1.2}
    \noindent
    문제에서 준 조건에 따라 phransal verb \textit{clean up}을 하나의 V로 간주하고자 한다. Conjunction이 같은 통사범주에 속한 성분끼리만 일어날 수 있다는 점을 고려하면 동사인 \textit{leash}와 \textit{clean up}은 주어진 문장 내에서 \textit{and}을 매개로 연결된 것으로 보인다. 그러나 이러한 분석은 한 가지 의문을 남긴다. 이들을 개별적인 동사구 핵으로 간주했을 때 남은 성분들이 이루는 \textit{after your pets}가 \textit{leash}와 \textit{clean up} 각각에 직접 후행하는 두 개의 문장을 구성하면 전자는 비문, 후자는 정문이 되기 때문이다. 한편 preposition을 제거한 형태인 \textit{your pets}가 \textit{leash} 뒤에 후행하면 이 문장은 정문이 된다. 이는 \textit{after your pet}이 V를 modify하는 PP로서 기능하는 것이 아니라 \textit{after}가 \textit{clean up}과 결합하여 하나의 phrasal verb [$_{\text{V}}$ clean up after]를 이루고 [$_{\text{NP}}$ your pets]가 이 phrasal verb와 [$_{\text{V}}$ leash] 각각의 목적어로 기능할 수 있음을 암시한다. 때문에 preposition이 PP의 head로만 나타날 수 있는 현재까지의 PSR로는 \textit{your pets}가 \textit{leash}와 \textit{after} 모두의 목적어가 되는 tree를 그려낼 수 없다.\\\textcolor{blue}{leash에 후행하는 NP를 생략하는 방법으로도 설명할 수 있음(인정답안)}
\end{spacing}
\subsubsection*{\textls[-50]{Q4  GPS10. Draw a Tree}}
\begin{forest}
    for tree = {s sep=20mm}
    [R[C[B]][D[F][E[G][H]]]]
\end{forest}
\subsubsection*{\textls[-50]{Q5  CPS2. Negative Polarity Items}}
\noindent
문제에서 준 자료 (a-f)를 bracketed diagram과 tree diagram으로 나타내면 다음과 같다.\\
\newline
\begin{spacing}{1.5}
\noindent
a) I didn't have any money.\\
\noindent
[$_{\text{TP}}$[$_{\text{NP}}$[$_{\text{N}}$I]][$_{\text{T}}$did][$_{\text{VP}}$[$_{\text{Neg}}$n't][$_{\text{V}}$have][$_{\text{NP}}$[$_{\text{D}}$any][$_{\text{N}}$money]]]]\\
\end{spacing}
\begin{forest}
    [TP[NP[N\\I]][T\\did][VP[Neg\\n't, name=src,tikz={\node [draw,ellipse,inner sep=-1pt, fit to=tree]{};}][V\\have][NP, name=tgt,tikz={\node [draw,ellipse,inner sep=-1pt, fit to=tree]{};}[D\\any][N\\money]]]]
    \draw[->,>=stealth] (src) to[out=north,in=north east] node[pos=0.5,yshift=3mm,xshift=15mm]{\textit{precedes, c-commands}} (tgt);
\end{forest}\\
\newline
\begin{spacing}{1.5}
\noindent
b) *I had any money.\\
\noindent
$^*$[$_{\text{TP}}$[$_{\text{NP}}$[$_{\text{N}}$I]][$_{\text{VP}}$[$_{\text{V}}$had][$_{\text{NP}}$[$_{\text{D}}$any][$_{\text{N}}$money]]]]\\
\end{spacing}
\begin{forest}
    [$^*$TP[NP[N\\I]][VP[V\\had][NP[D\\any][N\\money]]]]
\end{forest}\\
\newline
\begin{spacing}{1.5}
\noindent
\newpage
\noindent
c) I didn't read a single book the whole time I was in the library.\\
\noindent
[$_{\text{TP}}$[$_{\text{NP}}$[$_{\text{N}}$I]][$_{\text{T}}$did][$_{\text{VP}}$[$_{\text{Neg}}$n't][$_{\text{V}}$read][$_{\text{NP}}$a single book][$_{\text{AdvP}}$the whole time I was in the library]]]\\
\end{spacing}
\begin{forest}
    [TP[,phantom][NP,l=5mm[N\\I]][T\\did,l=5mm,before computing xy={s/.average={s}{siblings}}][VP, l=10mm[Neg\\n't,name=src,tikz={\node [draw,ellipse,inner sep=-1pt, fit to=tree]{};},l=10mm][,phantom][,phantom][V\\read,l=10mm][NP,l=10mm,name=tgt,tikz={\node [draw,ellipse,inner sep=-1pt, fit to=tree]{};} [a single book, roof]][AdvP,l=10mm[the whole time I was in the library., roof]]]]
    \draw[->,>=stealth] (src) to[out=south,in=south west] node[pos=0.5,yshift=-3mm,xshift=-10mm]{\textit{precedes, c-commands}} (tgt);
\end{forest}\\
\newline
\begin{spacing}{1.5}
\noindent
d) *I read a single book the whole time I was in the library.\\
$^*$[$_{\text{TP}}$[$_{\text{NP}}$[$_{\text{N}}$I]][$_{\text{VP}}$[$_{\text{V}}$read][$_{\text{NP}}$a single book][$_{\text{AdvP}}$the whole time I was in the library]]]\\
\end{spacing}
\begin{forest}
    [$^*$TP[NP[N\\I]][VP[,phantom][V\\read, l=10mm][NP,before computing xy={s/.average={s}{siblings}}, l=10mm [a single book, roof]][AdvP, l=10mm[the whole time I was in the library., roof]]]]
\end{forest}\\
\newline
\begin{spacing}{1.5}
\noindent
e) I did not have any money.\\
\noindent
[$_{\text{TP}}$[$_{\text{NP}}$[$_{\text{N}}$I]][$_{\text{T}}$did][$_{\text{VP}}$[$_{\text{Neg}}$not][$_{\text{V}}$have][$_{\text{NP}}$[$_{\text{D}}$any][$_{\text{N}}$money]]]]\\
\end{spacing}
\begin{forest}
    [TP[NP[N\\I]][T\\did][VP[Neg\\not,name=src,tikz={\node [draw,ellipse,inner sep=-1pt, fit to=tree]{};}][V\\have][NP,name=tgt,tikz={\node [draw,ellipse,inner sep=-1pt, fit to=tree]{};}[D\\any][N\\money]]]]
    \draw[->,>=stealth] (src) to[out=north,in=north east] node[pos=0.5,yshift=3mm,xshift=15mm]{\textit{precedes, c-commands}} (tgt);
\end{forest}\\
\newpage
\begin{spacing}{1.5}
\noindent
f) *Any money was not found in the box.\\
$^*$[$_{\text{TP}}$[$_{\text{NP}}$[$_{\text{D}}$Any][$_{\text{N}}$money]][$_{\text{T}}$was][$_{\text{VP}}$[$_{\text{Neg}}$not][$_{\text{V}}$found][$_{\text{PP}}$[$_{\text{P}}$in][$_{\text{NP}}$[$_{\text{D}}$the][$_{\text{N}}$box]]]]]\\
\end{spacing}
\begin{forest}
    [$^*$TP[NP,name=tgt,tikz={\node [draw,ellipse,inner sep=-1pt, fit to=tree]{};}[D\\Any][N\\money]][T\\was][VP[Neg\\not,name=src,tikz={\node [draw,ellipse,inner sep=-1pt, fit to=tree]{};}][V\\found][PP[P\\in][NP[D\\the][N\\box]]]]]
    \draw[dashed,gray,->,>=stealth] (src) to[out=south west,in=south] node[pos=0.5,yshift=-5mm,xshift=15mm]{\textit{does not precedes or c-commands}} (tgt);
\end{forest}\\
\newline
\noindent
이상의 주어진 자료를 근거로 다음과 같은 두 가지 가설을 수립할 수 있다.
\begin{tabbing}
00\=11 \=2 \kill
\>i)  \>Neg가 NPI에 precede해야 한다.\\
\>ii) \>Neg가 NPI를 (symmetrically) c-command해야 한다.\\
\end{tabbing}
이제 자료 (g)를 검토하여 두 가지 가설 중 하나를 채택하고자 한다.\\
\newline
\begin{spacing}{1.5}
\noindent
g) *The man that Susan didn't like had any money.\\
$^*$[$_{\text{TP}}$[$_{\text{NP}}$[$_{\text{D}}$The][$_{\text{N}}$man][$_{\text{CP}}$[$_{\text{C}}$that][$_{\text{TP}}$[$_{\text{NP}}$[$_{\text{N}}$Susan]][$_{\text{T}}$did][$_{\text{VP}}$[$_{\text{Neg}}$not][$_{\text{V}}$like]]]]][$_{\text{VP}}$[$_{\text{V}}$had][$_{\text{NP}}$[$_{\text{D}}$any][$_{\text{N}}$money]]]]\\
\end{spacing}
\begin{forest}
        [$^*$TP[NP[D\\The][N\\man][CP[C\\that][TP[NP[N\\Susan]][T\\did][VP[Neg\\not,name=src,tikz={\node [draw,ellipse,inner sep=-1pt, fit to=tree]{};}][V\\like]]]]][VP[V\\had][NP,name=tgt,tikz={\node [draw,ellipse,inner sep=-1pt, fit to=tree]{};}[D\\any][N\\money]]]]
        \draw[->,>=stealth] (src) to[out=south,in=east] node[pos=0.5,yshift=0mm,xshift=30mm]{\textit{precedes but does not c-command}} (tgt);
\end{forest}\\
\begin{spacing}{1.5}
\noindent
Neg가 NPI에 precede함에도 불구하고 이 문장은 비문이 된다. 이로부터 Neg가 NPI를 precede하는 것만으로는 충분치 않고 c-command하는 관계에 있어야 함을 확인하였다. 따라서 가설 (ii)가 유지된다.    
\end{spacing}
\newpage
\subsubsection*{\textls[-50]{Q6  CPS2. Binding Domain}}
\noindent
Binding domain의 정의와 현재까지 확인된 세 가지 Binding Principle은 다음과 같다.
\begin{tabbing}
    00000\=11 \=2 \kill
    x)\>\textit{\textbf{Binding Domain}}: The clause containing the NP.\\
    \\
    xii)\>\textit{\textbf{The Binding Principles}}\\
    \>\textit{Principle A}: An anaphor must be bound in its binding domain.\\
    \>\textit{Principle B}: A pronoun must be free in its binding domain.\\
    \>\textit{Principle C}: An R-expression must be free.\\
\end{tabbing}
Principle B에 의하면 pronoun은 그것이 포함된 (minimal) TP 안에서 bound되어서는 안 된다.\\
\newline
\textbf{Andy}$_{\text{i}}$ dismayed [\textbf{his}$_{\text{i}}$ father]$_{\text{m}}$.\\
\newline
\begin{spacing}{1.5}
\noindent
his$_{\text{i}}$는 주어인 Andy$_{\text{i}}$를 antecedent로 취하는 possessive pronoun이다. 그러나 이를 용어 그대로 pronoun으로 분류하면 Binding Principle에 의해 모순이 발생한다. Andy$_{\text{i}}$가 his$_{\text{i}}$를 bind하고 있는데 이는 pronoun이 binding domain에서 free여야 한다는 Principle B에 위배되며, 이 이론에 따르면 위 문장은 비문으로 판정되어야 한다. 이러한 모순은 현재의 이론이 possessive pronoun의 출현 조건을 밝히기에 충분치 않으며 binding 혹은 pronoun의 재정의가 필요함을 보여준다. \\
\end{spacing}
\subsubsection*{\textls[-50]{Q7  CPS4. Japanese}}
\noindent
a) Kazukowa$_{\text{i}}$ [$_{\text{CP}}$ [$_{\text{TP}}$ Tarooga$_{\text{k}}$ zibunzisino$_{\text{k/$^{*}$i}}$ hihansta][$_{\text{C}}$to]] itta.\\
\newline
\begin{spacing}{1.5}
\noindent
\textit{Question 1}: On the basis of only the data in (a) is \textit{zibunzisin} an anaphor or a pronoun? How can you tell?\\
문제에서 준 접사에 관한 조건에 의하면 Tarooga 는 zibunzisino을 c-command한다. 이 두 NP는 k로써 coindexed되어야만 문법성을 만족하며, zibunzisin의 binding domain 바깥에 있는 NP의 index i가 부여되면 비문이 된다. 즉 embedded clause 안에서 Tarooga는 zibunzisino를 bind하여야 한다. 만일 zibunzisin이 pronoun이라면 이는 Principle B에 위배되는 반면 zibunzisin이 anaphor라면 주어진 구조는 Principle A를 준수하여 정문이 된다. 따라서 현재까지의 기준으로 미루어 볼 때 zibunzisin은 anaphor에 해당하는 것으로 보인다.\\
\end{spacing}
\noindent
b) Kazukowa$_{\text{i}}$ [$_{\text{CP}}$ [$_{\text{TP}}$ zibunzisinga$_{\text{i}}$ Tarooo korosita] [$_{\text{C}}$  to]] omotteiru.\\
\newline
\begin{spacing}{1.5}
\noindent
\textit{Question 2}: Given this additional evidence, do you need to revise your hypothesis from question 1? Is \textit{zibunzisin} an anaphor, a pronoun or something else entirely? How can you tell?\\
(b)에서는 zibunzisinga가 binding domain--embedded clause 안--에서 free이다. 때문에 zibunzisin이 anaphor의 지위를 갖추고 있다고 보면 이는 Principle A에 위배된다. 또한 zibunzisin은 embedded clause 바깥의 NP Kazukowa$_{\text{i}}$ 와 coindexed되었으므로, 같은 개체를 binding domain 바깥의 NP와 corefer하는 것을 허용하는 pronoun의 성격을 띄고 있다. 이는 (a)에서의 관찰과 충돌하며 현재까지의 이론으로는 zibunzisin이 anaphor와 pronoun 중 어느 하나라고 단정 지을 수 없게 된다.\\
\textcolor{red}{“그럼 무엇?”: zibunzisin이 주어로 있느냐 목적어로 있느냐에 따라 경우를 구분지어 역할이 바뀐다는 사실을 구체적으로 서술했어야 함. (-2)\\}
\end{spacing}
\noindent
c) $^*$Kazukowa$_{\text{i}}$ [$_{\text{CP}}$ [$_{\text{TP}}$ zibunzisinga$_{\text{k}}$ Tarooo$_{\text{k}}$ korosita] [$_{\text{C}}$  to]] omotteiru.\\
\newline
\begin{spacing}{1.5}
\noindent
\textit{Question 3}: Sentence (c) is a violation of which binding principle? (A, B, or C?) Which NP is binding which other NP in this sentence to cause the ungrammaticality?\\
Embedded clause 안의 NP Tarooo는 현실의 인물을 지칭하는 R-expression으로서 Principle C에 의해 조건 없이 free여야 한다. 그러나 문제에서 준 접사 조건에 의하면 zibunzisinga는 Tarooo를 c-command 하며 이 두 NP는 k로 coindexed되어 있다. 따라서 zibunzisinga가 Tarooo를 bind함으로써 (c)는 Principle C를 위배, 비문이 된다.\\
\end{spacing}
% --------------------------------------------------------------
%   name=src,tikz={\node [draw,ellipse,inner sep=-1pt, fit to=tree]{};}
%   name=tgt,tikz={\node [draw,ellipse,inner sep=-1pt, fit to=tree]{};}
%   \draw[->,>=stealth] (src) to[out=north,in=north east] node[pos=0.5,yshift=3mm,xshift=15mm]{\textit{modifies}} (tgt);  
% --------------------------------------------------------------
%     You don't have to mess with anything below this line.
% --------------------------------------------------------------
 
\end{document}